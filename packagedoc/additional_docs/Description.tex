% Copyright 2020-2024 Robert Bosch GmbH

% Licensed under the Apache License, Version 2.0 (the "License");
% you may not use this file except in compliance with the License.
% You may obtain a copy of the License at

% http://www.apache.org/licenses/LICENSE-2.0

% Unless required by applicable law or agreed to in writing, software
% distributed under the License is distributed on an "AS IS" BASIS,
% WITHOUT WARRANTIES OR CONDITIONS OF ANY KIND, either express or implied.
% See the License for the specific language governing permissions and
% limitations under the License.

\hypertarget{description-get-robotframework-xml-result}{%
\section{Get \rfwcore\ XML result}
\label{description-get-robotframework-xml-result}}

In order to manage test cases and their results Rational Quality Manager (RQM), 
certain traceable information, such as version, test case ID, component, etc., 
is required. 

This enables the \pkg\ tool to associate the imported test results with 
specific elements (Test Case) or link them to other entities (Build Record, 
Test Environment) in RQM.

These information should be provided in\rcode{Metadata}(for the whole
testsuite/execution info: version, build, ...) and\rcode{{[}Tags{]}}information
(for specific test case info: component, test case ID, requirement ID, ...) of
\rfwcore\ test case.
Then when executing \rfwcore\ test case(s), the generated \rfwcore\ result file 
(default is \emph{output.xml}) will contain all of them and ready for importing.

Sample \rfwcore\ test case with the neccessary information for importing to RQM:

\begin{robotcode}[caption=Sample \rfwcore\ test case,
                  linebackgroundcolor=\hlcode{2,3,4}]
*** Settings ***
Metadata   project      ROBFW             # Test Environment on RQM for linking
Metadata   version_sw   SW_VERSION_0.1    # Build Record on RQM for linking
Metadata   machine      %{COMPUTERNAME}   # Hostname attribute in RQM Test Case Result
Metadata   component    Import_Tools      # Component attribute in RQM Test Case
Metadata   team-area    Internet Team RQM  # team-area (case-sensitive) on RQM for linking

*** Test Cases ***
Testcase 01
   [Documentation]   This test is traceable with provided tcid
   [Tags]   TCID-1001   FID-112   FID-111    robotfile-https://github.com/test-fullautomation
   Log      This is Testcase 01

Testcase 02
   [Documentation]  This new test case will be created if -createmissing argument
               ...  is provided when importing
   [Tags]   FID-113  robotfile-https://github.com/test-fullautomation
   Log      This is Testcase 02
\end{robotcode}

\begin{boxhint} {Hint}
  In case you are using \rfw, above highlighted\rcode{Metadata} definitions are 
  not required because they have been handled by
  \href{https://github.com/test-fullautomation/robotframework-testsuitesmanagement}
  {RobotFramework\_TestsuitesManagement} library within\rcode{Suite Setup}.
\end{boxhint}

\hypertarget{description-tool-features}{%
\section{Tool features}\label{description-tool-features}}

After getting the \rfwcore\ \emph{*.xml} result file(s), you can use the \pkg\ tool to
import them into RQM.

Its usage and features are described as following sections.

\subsection{Usage}
Use below command to get tools's usage:
\begin{robotlog}
RobotLog2DB -h
\end{robotlog}

The tool's usage should be showed as below:
\begin{robotlog}
usage: RobotLog2RQM (RobotXMLResult to RQM importer) [-h] [-v] [--recursive]
                    [--createmissing] [--updatetestcase] [--dryrun]
                    resultxmlfile host project user password testplan

RobotLog2RQM imports XML result files (default: output.xml) generated by the
                  Robot Framework into an IBM Rational Quality Manager.

positional arguments:
resultxmlfile     absolute or relative path to the xml result file
                  or directory of result files to be imported.
host              RQM host url.
project           project on RQM.
user              user for RQM login.
password          password for RQM login.
testplan          testplan ID for this execution.

optional arguments:
-h, --help        show this help message and exit
-v, --version     Version of the RobotLog2RQM importer.
--recursive       if set, then the path is searched recursively for
                  log files to be imported.
--createmissing   if set, then all testcases without tcid are created
                  when importing.
--updatetestcase  if set, then testcase information on RQM will be updated
                  bases on robot testfile.
--dryrun          if set, then verify all input arguments
                  (includes RQM authentication) and show what would be done.
\end{robotlog}

As above instruction, \pkg\ tool requires 5 positional arguments consists of:
\begin{itemize}
  \item The \rfwcore\ result file/folder\rlog{resultxmlfile}
  \item The RQM authentication\rlog{host},\rlog{project},\rlog{user},\rlog{password}
  \item The RQM\rlog{testplan}ID which will contains all importing test results
\end{itemize}

\subsection{Basic import feature}
Use the below command for the simple import the \emph{output.xml} file to RQM
project \textbf{ROBFW-AIO} which is hosted at
\href{https://sample-rqm-host.com}{https://sample-rqm-host.com}
\begin{robotlog}
RobotLog2RQM output.xml https://sample-rqm-host.com ROBFW-AIO test_user test_pw 720
\end{robotlog}

When command is executed, the tool will process with following steps:
\begin{itemize}
\item Login the RQM server with the provided credential, tehn verify the
      existences of given\rlog{project},\rlog{testplan} on RQM
\item Create RQM \textbf{Build Record} and \textbf{Test Environment} (if already
      provided in \rfwcore\ test case and not existing on RQM)
\item Create new RQM \textbf{Test Case Execution Record - TCER}
      (if it is not existing) bases on test case ID (defined \rcode{TCID-xxx} in
      \rcode{[Tags]} of \rfwcore\ Test Cases) and \rlog{testplan} ID
\item Create new RQM \textbf{Test Case Execution Result} which contents
      the detail and result state of \rfwcore\ test case
\item Link all test case(s) to provided \rlog{testplan}
\end{itemize}

\subsection{Verify the given arguments}
In case you just want to verify whether the given \emph{*.xml} file/folder and
the RQM authentication in arguments are corrected or not, the optional argument
\rlog{--dryrun} will help to do it.
\\
In the dryrun mode, \pkg\ will not create any resources on RQM, it
just verify:
\begin{itemize}
  \item The given \rfwcore\ result file/folder is valid or not
  \item The given RQM authentication is correct or not
  \item The given RQM project and testplan are existing or not
\end{itemize}

\subsection{Import multiple *.xml result files}
\pkg\ accepts the first argument \rlog{resultxmlfile} can be a single file or
the folder that contains multiple \rfwcore\ result files.

When the folder is used, \pkg\ will only search for \emph{*.xml} file under
given directory and exclude any file within subdirectories as default.

In case you have result file(s) under the subdirectory of given folder and want
these result files will also be imported, the optional argument
\rlog{--recursive} should be used when executing \pkg\ command.

When \rlog{--recursive} argument is set, \pkg\ will walk through the given
directory and its subdirectories to discover and collect all available
\emph{*.xml} for importing.

For example: your result folder has a structure as below:

\begin{robotlog}
logFolder
   |_____ result_1.xml
   |_____ result_2.xml
   |_____ subFolder_1
   |         |________ result_sub_1.xml
   |         |________ subSubFolder
   |                       |______ result_sub_sub_1.xml
   |_____ subFolder_2
             |________ result_sub_2.xml
\end{robotlog}

\begin{itemize}
\item Without \rlog{--recursive}: only \textbf{result\_1.xml} and
      \textbf{result\_2.xml} are found for importing.
\item With \rlog{--recursive}: all \textbf{result\_1.xml},
      \textbf{result\_2.xml}, \textbf{result\_sub\_1.xml},
      \textbf{result\_sub\_2.xml} and \textbf{result\_sub\_sub\_1.xml} will
      be imported.
\end{itemize}

\subsection{Create missing Test Case on RQM}
By default, \pkg\ tool will not touch (create, update) any RQM Test Case.

If the \rcode{TCID-xxx} information is missing in \rcode{[Tags]} section of 
\rfwcore\ Test Case, an error message will be raised for that specific importing 
Test Case, and the tool will proceed with the next Test Cases accordingly
\begin{pythonlog}
ERROR: There is no 'tcid' information for importing test 'Testcase 01'.
\end{pythonlog}

So that, in order to import those missing \rcode{TCID-xxx} Test Cases, the
optional arguments \rlog{--createmissing} should be provided in the \pkg\
arguments.

When \rlog{--createmissing} is used, \pkg\ will help to create RQM Test Cases
bases on the defined information in \rfwcore\ Test Cases.
It obtains the new Test Case ID and uses it for linking to related RQM resources 
\textbf{TCER}, \textbf{Test Case Execution Result} as basic feature.

The new IDs for the created Test Cases are also displayed in the execution log.
You can copy these IDs and update the \rcode{TCID-xxx} information in \rfwcore\ 
Test Cases for the next execution. This information will then be available in 
the generated \emph{*.xml} result file for importing.

Please refer \hyperref[description-robot-testcase-information-on-rqm]
{\rfwcore\ Test Case Information on RQM} section for for details on how the 
defined information in \rfwcore\ Test Cases is reflected in RQM.

\subsection{Update existing Test Case on RQM}
In case the Test Case is existing on RQM, but you want to update its attribute(s)
such as \textbf{Component}, \textbf{Description}, ... the optional argument
\rlog{--updatetestcase} should be used.

\pkg\ will update RQM Test Case resource bases on the defined information in
\rfwcore\ Test Case before creating its result.

Please refer \hyperref[description-robot-testcase-information-on-rqm]{\rfwcore\ Test Case Information on RQM}
section for for details on how the defined information in \rfwcore\ Test Cases 
is reflected in RQM.

\hypertarget{description-robot-testcase-information-on-rqm}{%
\section{\rfwcore\ Test Case Information on RQM:}
\label{description-robot-testcase-information-on-rqm}}
For more detail about the mapping between the defined information from \rfwcore\ 
Test Case to \rfwcore\ result (\emph{output.xml}) file and their reflections on
RQM WebApp, please refer below mapping table:

\begin{table}[h]
   \label{table:RobotLog2RQM-mapping}
   \begin{tabular}{|p{0.115\linewidth}|p{0.12\linewidth}|p{0.31\linewidth}|p{0.45\linewidth}|}
      \hline
      \multicolumn{2}{|c|}{\textbf{RQM data}}
                                      &\multicolumn{2}{|c|}{\textbf{\rfwcore}}\\
      \hline
      \textbf{Resource} &\textbf{Attribute/ Field}
                                      &\textbf{Testsuite/Testcase}
                                                    &\textbf{Output.xml}\\
      \hline
      Build Record      &Title        &\rcode{Metadata version_sw Build}
                                                    &\textbf{//suite/metadata/item[@name="version\_sw"]}\\
      \hline
      Test Environment  &Title        &\rcode{Metadata project Environment}
                                                    &\textbf{//suite/metadata/item[@name="project"]}\\
      \hline
      \multirow{8}{*}{\textbf{Test Case}}
                        &ID           &\rcode{[Tags] tcid-xxx}
                                                    &\textbf{//suite/test/tags/tag[@text="tcid-xxx"]}\\
                        \cline{2-4}
                        &Name         &tesname      &\textbf{//suite/test/@name}\\
                        \cline{2-4}
                        &Team Area    &\rcode{Metadata team-area Team_Area}
                                                    &\textbf{//suite/metadata/item[@name="team-area"]}\\
                        \cline{2-4}
                        &Description  &test doc - \rcode{[Documentation]}
                                                    &\textbf{//suite/test/doc/@text}\\
                        \cline{2-4}
                        &Owner        &provided \rlog{user} in cli
                                                    &\\
                        \cline{2-4}
                        &Component/ Categories
                                      &\rcode{Metadata component Component}
                                                    &\textbf{//suite/metadata/item[@name="component"]}\\
                        \cline{2-4}
                        &Requirement ID
                                      &\rcode{[Tags] fid-yyy}
                                                    &\textbf{//suite/test/tags/tag[@text="fid-yyy"]}\\
                        \cline{2-4}
                        &Robot File   &\rcode{[Tags] robotfile-zzz}
                                                    &\textbf{//suite/test/tags/tag[@text="robotfile-zzz"]}\\
      \hline
      \multirow{8}{\linewidth}{\textbf{Test Case Execution Record (TCER)}}
                        &Owner        &provided \rlog{user} in cli
                                                    &\\
                        \cline{2-4}
                        &Team Area    &\rcode{Metadata team-area Team_Area}
                                                    &\textbf{//suite/metadata/item[@name="team-area"]}\\
                        \cline{2-4}
                        &Test Plan    &Interaction URL to provided \rlog{testplan} in cli
                                                    &\\
                        \cline{2-4}
                        &Test Case    &Interaction URL to provided test case ID:
                                      provided tcid in \rcode{[Tags]: tcid-xxx} or
                                      generated tcid when using \rlog{-createmissing}
                                                    &\textbf{//suite/test/tags/tag[@text="tcid-xxx"]}\\
                        \cline{2-4}
                        &Test Environment
                                      &\rcode{Metadata project Environment}
                                                    &\textbf{//suite/metadata/item[@name="project"]}\\
      \hline
      \multirow{13}{\linewidth}{\textbf{Test Result }}
                        &Owner        &provided \rlog{user} in cli
                                                    &\\
                        \cline{2-4}
                        &Tested By    &provided \rlog{user} in cli - userid must be used
                                                    &\\
                        \cline{2-4}
                        &Team Area    &\rcode{Metadata team-area Team_Area}
                                                    &\textbf{//suite/metadata/item[@name="team-area"]}\\
                        \cline{2-4}
                        &Actual Result&Test case result (\ifpassed{PASSED}, \iffailed{FAILED}, \ifunknown{UNKNOWN})
                                                    &\textbf{//suite/test/status/@status}\\
                        \cline{2-4}
                        &Host Name    &\begin{robotcode}
  Metadata machine %{COMPUTERNAME}
\end{robotcode}
                                                    &\textbf{//suite/metadata/item[@name="machine"]}\\
                        \cline{2-4}
                        &Test Plan    &Interaction URL to provided \rlog{testplan} in cli
                                                    &\\
                        \cline{2-4}
                        &Test Case    &Interaction URL to provided test case ID:
                                      provided tcid in \rcode{[Tags]: tcid-xxx} or
                                      generated tcid when using \rlog{-createmissing}
                                                    &\textbf{//suite/test/tags/tag[@text="tcid-xxx"]}\\
                        \cline{2-4}
                        &Test Case Execution Record
                                      &Interaction URL to TCER ID
                                                    &\\
                        \cline{2-4}
                        &Build        &\rcode{Metadata version_sw Build}
                                                    &\textbf{//suite/metadata/item[@name="version\_sw"]}\\
                        \cline{2-4}
                        &Start Time   &Test case start time
                                                    &\textbf{//suite/test/status/@starttime}\\
                        \cline{2-4}
                        &End Time     &Test case end time
                                                    &\textbf{//suite/test/status/@endtime}\\
                        \cline{2-4}
                        &Total Run Time
                                      &Calculated from start and end time
                                                    &\\
                        \cline{2-4}
                        &Result Details
                                      &Test case message log
                                                    &\textbf{//suite/test/status/@text}\\
                        \cline{2-4}
      \hline
   \end{tabular}
   \caption{RQM data \& \rfwcore\ }
\end{table}
